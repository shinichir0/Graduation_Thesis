\chapter{Hierarchical Temporal Memory(HTM)}
\section{HTMの概要}
\begin{itemize}
  \item HTMは大脳皮質の構造を模したニューラルネットワーク
  \item 特徴は並列同時予測
\end{itemize}

\section{HTMの構造}
\subsection{セルの状態変化}
\subsection{カラム構造}
\subsection{パターン表現}
\subsection{疎な分散表現}


\section{HTMの学習アルゴリズム}
\subsection{セグメント集合を用いた接続値の更新}


\section{HTMの問題点}
\begin{itemize}
  \item 疎な分散表現を用いたために発火するセルが徐々に少なくなり消失する。
  \item 学習が大きく進んだパターンにおいて表現が疎になった時に次のパターンに繋がっていたセルが消失するために学習が損失
\end{itemize}
