\chapter{結論}
本論文では、HTMを複数前の時刻におけるパターンを表現するセルとの接続も持つように改良することを提案した。
HTMの構造の面では接続セグメントに時間軸を導入した。
またHTMの学習アルゴリズムの面では、シナプス接続において複数時刻に渡るセルの発火との関係を持つことと予測状態のセルの計算において複数時刻からの繋がりの重ね合わせを行うことで、長期依存関係を保持することに成功した。
これによって並列予測を要する時系列データに対しての予測タスクにおいて、従来のHTMよりも高い精度を記録することが確認された。

今後の展望として、分散表現をHTMのカラム表現に変換する方法を確立することと、単語分散表現を用いることで自然言語処理における様々な言語モデルに適応することを考えている。
これによって脳の言語処理における語彙を司る分野と文脈を司る分野に分けた学習を模した言語学習が可能になると考えられる。
