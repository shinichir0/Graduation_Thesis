\chapter{評価実験}
ここはあとで綺麗に書きます。

\section{実験1}
\subsection{実験目的}
適切なセグメント集合のサイズ検定

\subsection{実験概要}
セグメント集合のサイズをテンソルの合計の大きさを変えずに様々な条件で検定する。

\subsection{実験条件}
セグメント集合のサイズ[カラム数、セル数、セグメント数、時間軸長]
$セグメント集合のテンソルの合計の大きさは(カラム数^2*セル数^2*セグメント数*時間軸)
従来型HTMのテンソルの合計サイズ(512,16,32,1)の場合512^2*16^2*32*1=2^31$

\begin{itemize}
  \item 512,16,4,8
  \item 512,16,8,4
  \item 512,8,16,8
  \item 512,8,8,16
\end{itemize}

\subsection{実験結果}

\section{実験2}
\subsection{実験目的}
改良型HTMにおける長期依存考慮の性能検定

\subsection{実験概要}
様々な合成関数を学習させた提案モデルを従来型のHTMとLSTM、GRUを比較

\subsection{実験結果}

\begin{itemize}
  \item 改良型HTM
  \item HTM
  \item LSTM
  \item GRU
\end{itemize}

\section{実験3}
\subsection{実験目的}
改良型HTMにおける同時並行予測を必要とする長期依存考慮の性能検定

\subsection{実験概要}
様々な合成関数を学習させた提案モデルを従来型のHTMとLSTM、GRUを比較

\subsection{実験結果}

\begin{itemize}
  \item 改良型HTM
  \item HTM
  \item LSTM
  \item GRU
\end{itemize}
