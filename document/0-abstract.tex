\abstract
本論文では大脳皮質の構造と学習アルゴリズムを模した時系列予測モデルであるHierarchical Temporal Memory(HTM)に対して、長期依存関係を考慮できるような改良案を提案する。
HTMは脳の神経細胞を表すセルを2次元マップ上に配置し、活性化状態になったセルの集合によりデータを表現する。
また予測状態になったセルの集合により次の時刻のデータを予測する。
各セル間を繋げるシナプス接続をヘブ則に基づいて更新することでデータ同士の関係を保持し、時系列データを学習する。
従来のモデルでは時系列データ中の各データに対して一時刻前のデータとの接続のみを学習していたが、提案モデルでは複数時刻前のデータとの接続を学習させることを目的とした。
具体的には構造と学習アルゴリズムの両面で従来のHTMを改良したモデルを提案する。
構造ではシナプスの集まりであるセグメントに対して時間軸を導入した。
学習アルゴリズムではセルの予測状態への遷移において複数時間前のデータとの接続を利用する。
実験により提案モデルは時系列データの予測タスクにおいて従来モデルよりも高い精度を記録することを確認した。
